\documentclass[english]{article}
\usepackage[T1]{fontenc}
\usepackage[utf8]{inputenc}
\usepackage{babel}
\usepackage[unicode=true,pdfusetitle,
 bookmarks=true,bookmarksnumbered=false,bookmarksopen=false,
 breaklinks=true,pdfborder={0 0 1},backref=false,colorlinks=false]
 {hyperref}
\usepackage{tabularx}
\usepackage{graphicx}
\graphicspath{{images/}}
\usepackage{svg}
\usepackage{float}
\usepackage{titling}
\renewcommand{\arraystretch}{1.4}
\newcommand{\code}[1]{\texttt{#1}}
\usepackage{listings}
\usepackage{color}
\definecolor{javared}{rgb}{0.6,0,0} % for strings
\definecolor{javagreen}{rgb}{0.25,0.5,0.35} % comments
\definecolor{javapurple}{rgb}{0.5,0,0.35} % keywords
\definecolor{javadocblue}{rgb}{0.25,0.35,0.75} % javadoc

\lstset{language=Java,
	basicstyle=\ttfamily\small,
	xleftmargin=0cm,
	keywordstyle=\color{javapurple}\bfseries,
	stringstyle=\color{javared},
	commentstyle=\color{javagreen},
	morecomment=[s][\color{javadocblue}]{/**}{*/},
	numbers=left,
	numberstyle=\tiny\color{black},
	stepnumber=1,
	numbersep=10pt,
	tabsize=4,
	showspaces=false,
	showstringspaces=false,
	linewidth=13cm,
	breaklines=true
	}

\pretitle{%
	\begin{center}
		\LARGE
		\includegraphics[width=250pt]{../other/Logo_blu.png}\\[\bigskipamount]~\\[\bigskipamount]
	}
\posttitle{\end{center}}

\begin{document}

\title{Politecnico di Milano\\
 A.A. 2016–2017 \\
Software Engineering 2: “PowerEnJoy” \\
\emph{Code Inspection Document}}

\author{Pietro Ferretti, Nicole Gervasoni, Danilo Labanca}
\date{February 5, 2017}
\maketitle

\newpage

\tableofcontents{}

\newpage

\section{Assigned Class}
% <state the namespace pattern and name of the classes that were assigned to you.>

\texttt{apache-ofbiz-16.11.01/framework/service/\\
\hspace*{1cm}src/main/java/org/apache/ofbiz/service/job/JobManager.java}

\section{Functional Role}% of assigned set of classes: <elaborate on the functional role you have identified for the class cluster that was assigned to you, also, elaborate on how you managed to understand this role and provide the necessary evidence, e.g., javadoc, diagrams, etc.>

\paragraph{Javadoc}

\begin{lstlisting}

/**
* Job manager. The job manager queues and manages jobs. Client code can queue a job to be run immediately
* by calling the runJob({@link #runJob(Job)}) method, or schedule a job to be run later by calling the
* {@link #schedule(String, String, String, Map, long, int, int, int, long, int)} method.
* Scheduled jobs are persisted in the JobSandbox entity.
* <p>A scheduled job's start time is an approximation - the actual start time will depend
* on the job manager/job poller configuration (poll interval) and the load on the server.
* Scheduled jobs might be rescheduled if the server is busy. Therefore, applications
* requiring a precise job start time should use a different mechanism to schedule the job.</p>
*/
\end{lstlisting}

\section{List of Issues} % found by applying the checklist: <report the classes/code fragments that do not fulfill some points in the check list. Explain which point is not fulfilled and why.>

\subsection{Naming Conventions}

\paragraph{1. Meaningful Names}
\paragraph{}
sono tutti buoni

\paragraph{2. One-character variables}

Ok.
There are no one-character variables.

\paragraph{3. Class names}
Ok.
Every class name is in mixed case and properly capitalized.

\paragraph{4. Interface names}
Ok.
No interfaces are declared.
(se ce ne sono) every interface used by the code is in mixed case and properly capitalized.

\paragraph{5. Method names}
Ok.
Every method name is a verb.
Every method name is camelCase and properly capitalized.

\paragraph{6. Class variables}
Ok.
Every class variable is in mixed case and properly capitalized.

\paragraph{7. Constants}
NO.
%https://stackoverflow.com/a/30789385
module and istanceId are immutable, so they can be considered constant. They should be capitalized.
registeredManagers is fine because it's mutable

\subsection{Indentation}
\paragraph{8. Number of spaces}
Ok.
The code is consistently indented with 4 spaces.

\paragraph{9. No tabs for indentation}
Ok.
No tabs are used to indent the code.

\subsection{Braces}
\paragraph{10. Consistent bracing style}
Ok.
The code is consistently braced following the \textit{"Kernighan and Ritchie"} style.

\paragraph{11. One-line statements bracing}
NO
"if" riga 326, 351, 354

\subsection{File Organization}
\paragraph{12. Blank lines as separation}
Ok.
Blank lines are present between each method, around imports and variable declarations.
Most of the methods also begin with a Javadoc.

\paragraph{13. Where practical, line length under 80 characters}
NOPE
righe 73, 74, 89, 126, 147, 150, 154, 156, 161, 182, 186-190, 195, 198, 201, 
A great number of lines exceed 80 characters

\paragraph{14. Line length always under 120 characters}
NEPPURE
righe 74, 186, 198, 217, 221, 222, 261-264, 273, 311, 315, 317, 387, 409, 429, 453, 498, 543, 560, 561
le dichiarazioni dei metodi sono lunghissime e wrappate poco

\subsection{Wrapping Lines}
\paragraph{15. Line breaks after commas and operators}
NO
riga 152, la virgola dovrebbe stare sopra % e chissenefrega

\paragraph{16. Higher-level breaks are used}
Ok.
Non ci sono line break con operatori

\paragraph{17. Statements are aligned to previous ones}
Ok.
Sì, per tutti

\subsection{Comments}

\subsection{Java Source Files}
\paragraph{20. Single public class or interface}
Ok.
Job manager is the only public class declared in the file.
There are no other classes.

\paragraph{21. The public class is the first class in the file}
Ok.
Job manager is the only public class declared in the file.
There are no other classes.

\paragraph{22. External program interfaces are consistent with the Javadoc}
Ok
abbiamo vari metodi pubblici:
getter:
- getDelegator
- getDispatcher
- getInstance
- getPoolState
poi altre robe
- isAvailable
- reloadCrashedJobs
- runJob
- schedule di tutti i tipi

la Javadoc parla di runJob e schedule

\paragraph{23. The Javadoc is complete}
NO.
\begin{itemize}
	\item No javadoc for 'module'! line 71
	\item No javadoc for 'instanceId'! line 71
	\item No javadoc for reloadCrashedJob!! line 304
	\item Missing @return tag on getInstance, line 88
	\item Missing @return tag on getDelegator, line 119
	\item Missing @return tag on getDispatcher, line 124
	\item Missing @param tag for 'limit' on poll, line 174
	\item Missing @return tag on poll, line 174
	\item Missing @param tag for 'job' on runJob, line 363
	\item Missing @throws tag for 'JobManagerException' on runJob, line 363
	\item Missing @throws tag for 'JobManagerException' on schedule, line 386, 408, 428, 453, 469, 498, 543
\end{itemize}

assertIsRunning, getRunPools sono private quindi non hanno necessariamente bisogno di javadoc

\subsection{Package and Import Statements}
\paragraph{24. Package statements are first, import statements second}
Ok.
One package statements.
All import statements immediately follow.

\subsection{Class and Interface Declarations}
\paragraph{25. The class declarations should follow a specific order}
- javadoc ok
- class declaration ok
- altri commenti /
- static variables ok
 - public ok
 - private ok
- normal variables 
- constructors
- methods

no, abbiamo variabili statiche, poi un po' di metodi statici, poi variabili normali, poi costruttori (getInstance è un costruttore), setter e getter poi un metodo statico (ma private!!)


\paragraph{26. Methods are grouped by functionality}
Ok

assertIsRunning
getInstance
shutDown

getDelegator
getDispatcher
getPoolState

isAvailable
getRunPools
pool
reloadCrashedJobs
runJob
schedule

\paragraph{27. The code is free of duplicates, long methods, big classes, breaking encapsulation, and coupling and cohesion are adequate}
small class
duplicates? no
short methods
no breaking encapsulation

low/loose coupling -> ci sono un sacco di delegator e dispatcher
high cohesion -> tutti i metodi servono a runnare/queuare jobs

\subsection{Initialization and Declarations}


\subsection{Method Calls}


\subsection{Arrays}
\paragraph{No off-by-one errors in array indexing}
Ok.
The only indexing is with foreach, no off-by-one errors.

\paragraph{No out-of-bounds indexes}
Ok.
No number indexing.

\paragraph{Constructors are called when a new array item is desired}
Ok.
quali nuovi array? non ce ne sono

\subsection{Object Comparison}
\paragraph{Objects are compared with equals}
Ok.
There are no object comparisons.

\subsection{Output Format}
\paragraph{Displayed output is free of spelling and grammatical errors}
riga 156:
Debug.logWarning(e, "Exception thrown while check lock on JobManager : " + instanceId, module);
dovrebbe essere "while checking"

riga 182:
Debug.logWarning("Unable to locate DispatchContext object; not running job!", module);
dovrebbe essere "job:", come negli altri log di debug

\paragraph{Error messages are comprehensive and useful}
Sì

\paragraph{Output is formatted correctly in terms of line breaks and spacing}
No line breaks in outputs
Some debug outputs don't have a trailing space ~

\subsection{Computation, Comparisons and Assignments}

\subsection{Exceptions}
% catch(Throwable t) è giustificato perché non vogliamo che il processo si interrompa


\subsection{Flow of Control}
\paragraph{All switch cases are addressed with a break}
Ok, no switch statements.

\paragraph{All switch statements have a default branch}
Ok, no switch statements.

\paragraph{All loops are correctly formed, with appropriate initialization, increments and termination expressions}
Ok.
All for loops are foreach, everything is fine.
The while loop at line 219:
GenericValue jobValue = jobsIterator.next();
while (jobValue != null) \{
jobValue = jobsIterator.next();
tutto ok, l'iteratore va avanti finché non finiscono i valori, poi esce dal while

while a riga 275 uguale

a posto

\section{Other Problems} % you have highlighted: <list here all the parts
%of code that you think create or may create a bug and explain why.>
nah

\section{Effort Spent}
\begin{itemize}
	\item{Pietro Ferretti:  hours of work}
	\item{Nicole Gervasoni:  hours of work}
	\item{Danilo Labanca:  hours of work}
\end{itemize}


\section{Revisions}

\subsection{Changelog}
\begin{itemize}
	\item{CID v1.0, published on February 5, 2017}
\end{itemize}
\end{document}
