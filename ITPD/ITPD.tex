\documentclass[english]{article}
\usepackage[T1]{fontenc}
\usepackage[utf8]{inputenc}
\usepackage{babel}
\usepackage[unicode=true,pdfusetitle,
 bookmarks=true,bookmarksnumbered=false,bookmarksopen=false,
 breaklinks=true,pdfborder={0 0 1},backref=false,colorlinks=false]
 {hyperref}
\usepackage{tabularx}
\usepackage{graphicx}
\graphicspath{{images/}}
\usepackage{svg}
\usepackage{float}
\usepackage{titling}
\renewcommand{\arraystretch}{1.4}
\newcommand{\code}[1]{\texttt{#1}}
\usepackage{array}

\pretitle{%
	\begin{center}
		\LARGE
		\includegraphics[width=250pt]{../other/Logo_blu.png}\\[\bigskipamount]~\\[\bigskipamount]
	}
\posttitle{\end{center}}

\begin{document}

\title{Politecnico di Milano\\
 A.A. 2016–2017 \\
Software Engineering 2: “PowerEnJoy” \\
\emph{Integration Test Plan Document}}

\author{Pietro Ferretti, Nicole Gervasoni, Danilo Labanca}
\date{January 15, 2017}
\maketitle

\newpage

\tableofcontents{}

\newpage

\section{Introduction}

\subsection{Revision History}

\subsection{Purpose and Scope}

\paragraph{}
The purpose of this document is to provide a complete description of the integration testing plans for PowerEnjoy. It will focus on testing the proper behavior of the software by checking the 
interoperability between its components.
It is intended for developers and for any team member involved in the testing process.
%TODO inserire outline dei contenuti?

\paragraph{}
The aim of this project is to specify in detail a new digital management software for PowerEnJoy, a car-sharing service that employs electric cars only.

\paragraph{}
PowerEnjoy will offer a very valuable service to its users, letting them borrow cars to drive around the city freely, as an alternative to their own vehicles and public transport.
Among the advantages of using PowerEnJoy we can note being able to find available cars in any place that is served by our system and having dedicated spots to park in (namely, PowerEnJoy's power grid stations).
Furthermore, thanks to the fact that all the cars that we provide are electrically powered, PowerEnJoy is also very environmentally friendly.

%\paragraph{}
%PowerEnJoy's users, after registering, will be able to reserve, unlock and drive the cars our system will provide. Users will be charged per minute until they park the car in a safe area and end the ride.
%Users will be able to park their car temporarily and use it again later, or end their ride remotely.
%
%Our system will incentivize virtuous behaviour by offering several discounts if certain conditions are met (like charging a car at a power grid station).

\subsection{List of Definitions and Abbreviations}

\subsubsection{Definitions}

\begin{itemize}
\item{\textit{Guest}: a person that is not registered to the system.}
\item{\textit{User}: a person that is registered to the system. Users can log in to the system with their email or username and their password. Their first name, last name, date of birth, driving license ID are stored in the database.}
\item{\textit{Safe area}: a location where the user can park and leave the car. Users can end their ride and park temporarily only in these locations. The set of safe areas is predefined by the system.}
\item{\textit{Power grid station}: a place where cars can be parked and plugged in. While a car is plugged in a power grid station its battery will be recharged. Power grid stations are by definition safe areas.}
\item{\textit{Available car}: a car that is currently not being used by any user, and has not been reserved either. Available cars are in good conditions (not dirty nor damaged) and don’t have dead batteries.}
\item{\textit{Reservation}:
	\begin{itemize}
		\item{the operation of making a car reserved for a user, i.e. giving permission to unlock and use the car only for that user, forbidding reservations by other users.}
		\item{the time period between the moment a reservation is requested and the moment the user unlocks the car, or the reservation is canceled.}
	\end{itemize}
}
\item{\textit{Ride}: the time period from the moment a reserved car is unlocked to the moment the user notifies that he wants to stop using the car and closes all the doors. A ride doesn’t stop when a car is temporarily parked, but continues until the user chooses to leave the car definitely.}
%\item{\textit{Possession}: users that have reserved and unlocked a car are said to have possession of the car. While a user has possession of a car they are the only person that can drive it, lock or unlock it, and no other person can take possession of it until the user frees it. Users lose possession of a car when their ride ends.}
\item{\textit{Temporary parking}: the act of parking a car in a safe area and, after notifying the system, locking it and leaving it for a finite amount of time. The user that does this retains the right to use the car and can unlock it later to use it again.}
\item{\textit{Bill}: a record of the money owed by the user at the end of a ride.}
\item{\textit{Outstanding bill}: a bill that hasn’t been paid yet. }
\item{\textit{Suspended user}: a user that cannot reserve or use cars. Usually users are suspended because they have outstanding bills.}
\item{\textit{Payment method}: a way to transfer money from the user to the system. Our system will only accept credit cards and online accounts like Paypal.}
\item{\textit{Payment API}: an interface to carry out money transactions, offered by the external provider associated to the payment method used (e.g. a bank).}
%\item{\textit{CAN bus}: a vehicle bus standard designed to allow micro controllers and devices to communicate with each other.}
\end{itemize}

\subsubsection{Acronyms}
\begin{itemize}
\item{\textbf{DD}: Design Document}
\item{\textbf{RASD}: Requirements Analysis and Specification Document}
\item{\textbf{DB}: Database}
%\item{\textbf{CVV}: Card Verification Value}
\item{\textbf{DOB}: Date of birth}
\item{\textbf{PGS}: Power Grid Station}
\item{\textbf{GPS}: Global Positioning System}
%\item{\textbf{CAN bus}: Controller Area Network bus}
\end{itemize}

\subsubsection{Abbreviations}
\begin{itemize}
\item{\textbf{[Gx]}: Goal}
\item{\textbf{[RE.x]}: Functional Requirement}
\item{\textbf{[UC.x]}: Use Case}
\end{itemize}

\subsection{List of Reference Documents}

\begin{itemize}
	\item{Requirements analysis and specification document: “RASD.pdf”}
	\item{Design document: “DD.pdf”}
	\item{Project description document: “Assignments AA 2016-2017.pdf”}
	%% io aggiungerei il documento di esempio
\end{itemize}

\section{Integration Strategy}

\subsection{Entry Criteria}

Before starting the integration testing phase specific conditions concerning the whole project development must be met.
First of all, it is fundamental that the Requirements Analysis and Specification Document and the Design Document have been properly written and completed.

%TODO aggiungere percentuali completamento?
Regarding the code development it is only needed that the \textbf{INSERIRE PERCENTUALI} has been written. Furthermore, each component has to be successfully unit tested before being involved in the integration testing.

%  This is a required step in order to have a complete picture of the interactions between the different components of the system and of the functionalities they offer. Secondly, the integration process should start only when the estimated percentage of completion of every component with respect to its functionalities is: • 100% for the Data Access Utilities component • At least 90% for the Taxi Management System subsystem • At least 70% for the System Administration and Account Management subsystems • At least 50% for the client applications It should be noted that these percentages refer to the status of the project at the beginning of the integration testing phase and they do not represent the minimum completion percentage necessary to consider a component for integration, which must be at least 90%. The choice of having different completion percentages for the different components has been made to reflect their order of integration and to take into account the required time to fully perform integration testing.


\subsection{Elements to be Integrated}

\subsection{Integration Testing Strategy}

\subsection{Sequence of Component / Function Integration}
%TODO choose just one title based on previous point

\section{Individual Steps and Test Description}

\subsection{Resource Management Subsystem}

\subsubsection{Component 1 - Component 2}

%% se necessario possiamo usare longtable
\begin{center}
	\begin{tabular}{ | m{7cm} | m{7cm} | }
		\hline 
		\multicolumn{2}{|c|}{\textbf{method(arg1, arg2)}} \\
		\hline
		\multicolumn{1}{|c|}{\textit{Input}} & \multicolumn{1}{|c|}{\textit{Result}} \\
		\hline
		input1 & result1 \\
		\hline
		input2 & result2 \\
		\hline
	\end{tabular}
\end{center}

\subsection{Application Subsystem}

\subsection{Integration between subsystems}

\section{Tools and Test Equipment Required}

\section{Program Stubs and Test Data Required}


\section{Effort Spent}
\begin{itemize}
	\item{Pietro Ferretti:  hours of work}
	\item{Nicole Gervasoni:  hours of work}
	\item{Danilo Labanca:  hours of work}
\end{itemize}


\section{Revisions}

\subsection{Changelog}
\begin{itemize}
	\item{ITDP v1.0, published on January 15, 2017}
\end{itemize}
\end{document}
