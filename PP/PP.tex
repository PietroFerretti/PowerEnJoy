\documentclass[english]{article}
\usepackage[T1]{fontenc}
\usepackage[utf8]{inputenc}
\usepackage{babel}
\usepackage[unicode=true,pdfusetitle,
 bookmarks=true,bookmarksnumbered=false,bookmarksopen=false,
 breaklinks=true,pdfborder={0 0 1},backref=false,colorlinks=false]
 {hyperref}
\usepackage{tabularx}
\usepackage{graphicx}
\graphicspath{{images/}}
\usepackage{svg}
\usepackage{float}
\usepackage{titling}
\renewcommand{\arraystretch}{1.4}
\newcommand{\code}[1]{\texttt{#1}}
\usepackage{array}
\usepackage{caption}

\pretitle{%
	\begin{center}
		\LARGE
		\includegraphics[width=250pt]{../other/Logo_blu.png}\\[\bigskipamount]~\\[\bigskipamount]
	}
\posttitle{\end{center}}

\begin{document}

\title{Politecnico di Milano\\
 A.A. 2016–2017 \\
Software Engineering 2: “PowerEnJoy” \\
\emph{\textbf{Project Plan}}}

\author{Pietro Ferretti, Nicole Gervasoni, Danilo Labanca}
\date{January 21, 2017}
\maketitle

\newpage

\tableofcontents{}

\newpage

\section{Introduction}

\subsection{Purpose}
The purpose of this document is to provide a detailed analysis of the PowerEnjoy software development project in terms of required cost and time. It highlights the estimation of 
\begin{itemize}
\item project size, calculated using the \emph{Function Points approach} by IBM;
\item project cost and effort, calculated using the \emph{COCOMO II} by Boehm.
\end{itemize}
Given the previous information we elaborate a feasible schedule considering all the necessary activities in detail, thus the best resources' allocation on each one. The last section of the document focuses on handling all the possible risks that could be met during the whole process, from the requirements analysis to the final testing and deployment.


\subsection{Scope}
The aim of this project is to specify and design a new digital management software for PowerEnJoy, a car-sharing service that employs electric cars only.

\paragraph{}
PowerEnJoy will offer a very valuable service to its users, letting them borrow cars to drive around the city freely, as an alternative to their own vehicles and public transport.
Among the advantages of using PowerEnJoy we can note being able to find available cars in any place that is served by our system and having dedicated spots to park in (namely, PowerEnJoy's power grid stations).
Furthermore, thanks to the fact that all the cars that we provide are electrically powered, PowerEnJoy is also very environmentally friendly.

%\paragraph{}
%PowerEnJoy's users, after registering, will be able to reserve, unlock and drive the cars our system will provide. Users will be charged per minute until they park the car in a safe area and end the ride.
%Users will be able to park their car temporarily and use it again later, or end their ride remotely.
%
%Our system will incentivize virtuous behaviour by offering several discounts if certain conditions are met (like charging a car at a power grid station).

\newpage
\subsection{List of Definitions and Abbreviations}

\subsubsection{Definitions}

%\begin{itemize}
%%\item{\textit{Guest}: a person that is not registered to the system.}
%\item{\textit{User}: a person that is registered to the system. Users can log in to the system with their email or username and their password. Their first name, last name, date of birth, driving license ID are stored in the database.}
%\item{\textit{Safe area}: a location where the user can park and leave the car. Users can end their ride and park temporarily only in these locations. The set of safe areas is predefined by the system.}
%\item{\textit{Power grid station} or \textit{Charging station}: a place where cars can be parked and plugged in. While a car is plugged in a power grid station its battery will be recharged. Power grid stations are by definition safe areas.}
%\item{\textit{Available car}: a car that is currently not being used by any user, and has not been reserved either. Available cars are in good conditions (not dirty nor damaged) and don't have dead batteries.}
%\item{\textit{Reservation}:
%	\begin{itemize}
%		\item{the operation of making a car reserved for a user, i.e. giving permission to unlock and use the car only for that user, forbidding reservations by other users.}
%		\item{the time period between the moment a reservation is requested and the moment the user unlocks the car, or the reservation is canceled.}
%	\end{itemize}
%}
%\item{\textit{Ride}: the time period from the moment a reserved car is unlocked to the moment the user notifies that he wants to stop using the car and closes all the doors. A ride doesn't stop when a car is temporarily parked, but continues until the user chooses to leave the car definitely.}
%%\item{\textit{Possession}: users that have reserved and unlocked a car are said to have possession of the car. While a user has possession of a car they are the only person that can drive it, lock or unlock it, and no other person can take possession of it until the user frees it. Users lose possession of a car when their ride ends.}
%\item{\textit{Temporary parking}: the act of parking a car in a safe area and, after notifying the system, locking it and leaving it for a finite amount of time. The user that does this retains the right to use the car and can unlock it later to use it again.}
%\item{\textit{Bill}: a record of the money owed by the user at the end of a ride.}
%%\item{\textit{Outstanding bill}: a bill that hasn't been paid yet. }
%\item{\textit{Suspended user}: a user that cannot reserve or use cars. Usually users are suspended because they have outstanding bills that have not been paid.}
%\item{\textit{Payment method}: a way to transfer money from the user to the system. Our system will only accept credit cards and online accounts like Paypal.}
%\item{\textit{Payment API}: an interface to carry out money transactions, offered by the external provider associated to the payment method used (e.g. a bank).}
%\item{\textit{CAN bus}: a vehicle bus standard designed to allow micro controllers and devices to communicate with each other.}
%\end{itemize}

\subsubsection{Acronyms}
\begin{itemize}
\item \textbf{ITPD}: Integration Test Plan Document
\item{\textbf{DD}: Design Document}
\item{\textbf{RASD}: Requirements Analysis and Specification Document}
\item{\textbf{DB}: Database}
%\item{\textbf{CVV}: Card Verification Value}
%\item{\textbf{DOB}: Date of birth}
\item{\textbf{PGS}: Power Grid Station}
\item{\textbf{GPS}: Global Positioning System}
\item{\textbf{API}: Application Programming Interface}
\item{\textbf{ISDTN}: International Standard Date and Time Notation}
\item \textbf{EM}: Effort Multiplier
% anche tutti i cost driver, ecc.?
%\item{\textbf{CAN bus}: Controller Area Network bus}
\item {\textbf{FP}: Function Points}
\item \textbf{ILF}: Internal Logic File
\item \textbf{ELF}: External Logic File
\item \textbf{EI}: External Input
\item \textbf{EO}: External Output
\item \textbf{EQ}: External Inquiries
%\item \textit{DBMS}: Database Management System
%\item \textbf{ETA}: Estimated Time of Arrival
\item \textbf{UI}: User Interface
\end{itemize}

%\subsubsection{Abbreviations}
%\begin{itemize}
%\item{\textbf{[Gx]}: Goal}
%\item{\textbf{[RE.x]}: Functional Requirement}
%\item{\textbf{[UC.x]}: Use Case}
%\end{itemize}

\subsection{List of Reference Documents}

\begin{itemize}
	\item{Requirements analysis and specification document: “RASD.pdf”}
	\item{Design document: “DD.pdf”}
	\item{Integration testing document: “ITPD.pdf”}
	\item{Project description document: “Assignments AA 2016-2017.pdf”}
	\item{Example document: “Project planning example document.pdf”}
	\item{“COCOMO II -- Model Definition Manual”, version 2.1, 1995-2000, Center for Software Engineering, USC}
\end{itemize}

\section{Project size, cost and effort estimation}

% descrizione sezione


%% TABELLA
\begin{center}
	\begin{tabular}{ | p{6cm} | p{6cm} | }
		\hline 
		\multicolumn{2}{|c|}{\textbf{Pay a Bill}} \\
		\hline
		\multicolumn{1}{|c|}{\textit{Input}} & \multicolumn{1}{c|}{\textit{Result}} \\
		\hline
		A valid session token, a bill that needs to be paid and a valid payment method &  The transaction is carried out; if it succeeds the bill is marked as paid, otherwise returns failure. \\
		\hline
		A valid session token, a bill that needs to be paid and an ill-formed payment method & An exception is raised. \\
		\hline
		A valid session token and a bill that needs to be paid & The system uses the payment method saved for the user to carry out the transaction; if it succeeds the bill is marked as paid, otherwise returns failure. \\
		\hline
		A valid session token and a bill that has already been paid & An exception is raised. \\
		\hline
		A valid session token and a non-existent bill & An exception is raised. \\
		\hline
		An invalid session token and a bill & An exception is raised (bad authentication). \\
		\hline
	\end{tabular}
\end{center}

% tabella somma FP
\begin{table}[H]
	\centering
	\makebox[\textwidth][c]{
		\begin{tabular}{ |p{8cm}|m{2cm}|p{1cm}| }
			\hline
			\textbf{ELF} & \textbf{Complexity} & \textbf{FPs} \\
			\hline
			elf n1 & Low & 5 \\
			elf n2 & High & 10 \\
			elf n3 & medium & 7 \\
			\hline
			\multicolumn{2}{|l|}{Total} & \textbf{22} \\
			\hline
		\end{tabular}
	}
	\caption{asdfasdf}
\end{table}

% oppure
\begin{table}[H]
	\centering
	\makebox[\textwidth][c]{
		\begin{tabular}{ |p{8cm}|m{2cm}|p{1cm}| }
			\hline
			\textbf{ELF} & \textbf{Complexity} & \textbf{FPs} \\
			\hline
			elf n1 & Low & 5 \\
			elf n2 & High & 10 \\
			elf n3 & medium & 7 \\
			\hline
			\multicolumn{2}{r}{\textit{total}} & \multicolumn{1}{l}{\textbf{22}} \\
		\end{tabular}
	}
	\caption{ewerewr}
\end{table}

% tabella lista cost driver
\begin{table}[H]
	\centering
	\makebox[\textwidth][c]{
		\begin{tabular}{ |p{8cm}|m{3cm}|p{1cm}| }
			\hline
			\textbf{Cost Driver} & \textbf{Rating Level} & \textbf{EM} \\
			\hline
			Documentation match to life-cycle needs (DOCU) & Nominal & 1.00 \\
			\hline
			\multicolumn{2}{|l|}{Total} & \textbf{1.00} \\
			\hline
		\end{tabular}
	}
	\caption{I'm a table.}
\end{table}

\subsection{Size estimation: function points}
Function points are useful in expressing the amount of business functionality our software has to provide to a user and are used to compute an estimation of its size.
After been identified and categorized into one of five types: outputs, inquiries, inputs, internal files, and external interfaces, each functional requirement is then assessed for complexity and assigned a number of function points.\\
We based our computation on tables and values in \emph{COCOMO II Model Definition Manual v. 2.1}.

% tabelline complessità

\subsubsection{Internal Logic Files (ILFs)}
%quali internal logic files abbiamo?
%quali entità di dati abbiamo?

They are all kinds of data used and managed by the application in  order to offer the expected functions.\\
Data will be organize in the following tables in the DB:
\begin{itemize}
\item \textbf{user} : name, surname, username, password, dob, email, licenseID, cvv, cardNumber, accountStatus

\item \textbf{bill} : associatedLicense, total, date, rideID, carID, paymentStatus

\item \textbf{car} : model, plate, ID, available, issues

\item \textbf{report} :carID, description, associatedLicense, date

\item \textbf{safeArea} : latitude, longitude, ID


\item \textbf{PGS} : latitude, longitude, ID

\item \textbf{plug} : available, ID

\item \textbf{reservation} : ID, associatedLicense, carID, date, status

\item \textbf{ride} : ID, associatedLicense, associatedBill, date, status, ridingTime, carID

\end{itemize}
The software will operate directly on the previously listed data and with the tables generated from their relations between each other.\\
All this data are modeled in simple structures so their complexity can be considered low (referring to tables).

\begin{center}
$ FPs (ILF) = 7 \times 9 + 10= 73 $
\end{center}

\subsubsection{External Logic Files (ELFs)}

The situations in which our system demands external data is when it needs informations regarding geolocation or when it must guarantee the legal soundness of the Driving License.\\
In particularly:

\begin{itemize}
	\item \textbf{GraphHopper API}:
		\begin{itemize}
			\item{Given the string containing the address, the API returns a pair of float representing the coordinates of that location.}
			\item{Given two pair of coordinates, the API return a float representing the time within two position.}
		\end{itemize}
	\item \textbf{Eucaris API}:
		\begin{itemize}
			\item{Given name, surname, driving license ID and expiration date as string, the API returns a boolean value representing the correspondence with an existing driving license in Eucaris DB.}
		\end{itemize}
\end{itemize}

In the final analysis, as the involved data are string and number with restrained size, we can assess this logic files as low complexity.\\

\begin{center}
	\begin{tabular}{ | p{6cm} | p{6cm} | p{6cm} }
		\hline
		\multicolumn{1}{|c|}{\textit{ELF}} & \multicolumn{1}{c|}{\textit{Complexity}} & \multicolumn{1}{c|}{\textit{FPs}} \\
		\hline
		Reverse geocoding & Low & 5 \\
		\hline
		Isochrone distance & Low & 5\\
		\hline
		Driving Licenes legal soudness & Low & 5\\
		\hline
		\multicolumn{2}{|c|}{\textit{Total}} & \multicolumn{1}{c|}{15} \\
		\hline
	\end{tabular}
\end{center}

\subsubsection{External Inputs (EIs)}

PowerEnjoy offers a remarkable series of functionalities that required user's input.\\
In particularly:

\begin{itemize}

	\item{\textbf{Login}: this functionality demands only two strings as parameters, the username and the password, that will be compared with the ones stored in the DB. We can consider as a low complexity operation.}
	
	\item{\textbf{User update}: this functionality includes a collection of operations that allow to modify each aspect of user's profile. The input data are simple strings. Since the possibility are conspicuous and the different elaborations aren't basic and futhermore they interest several components, we can regard this functionality as an average complexity operation.}
	
	\item{\textbf{Pay bill} (automatically/manually): this is one of the most complex operations. It involves internal components and external APIs and it demands two numbers as input. Given the relevance of the operation and the parts interested, we can consider this as high complexity operation.}

	\item{\textbf{Create reservation}: this functionality requires as input the user ID and car ID, that we can consider simple inputs, while his complexity is due to the components involved. In fact at the initial moment of the creation, the application verifies if the user is suspended and after that is modified the availabilty field of the tuple representing the car and the car itself is put in order to be unlocked. Moreover it's inserted in the list of reservation made by the user this new one. Given that we classify the operation as having average complexity.}

	\item{\textbf{Cancel reservation}: the analysis made for the previous functionality is valid also for this one. The operations are comparable as elaboration and thanks to this the complexity of this functionality is average.}

	\item{\textbf{Start ride}: this action is very simple as it's demands only the user ID and the car ID and it tallies the duration of the ride. Made this consideration we assues that this functionality as low complexity.}

	\item{\textbf{End ride}: this functionality is the most complex one because involves various components and includes the payment functionality that has high complexity. When a ride ends the car is set as available in the DB, it creates a bill for the user and initiates the payment. As usual the inputs are simple strings.}

	\item{\textbf{Park}: this functionality expects as inputs the position of the car that are a pair of numbers, the user ID and the car ID. In the elaboration of this data the components interested are the internal car system and the DB to verify if the position belongs to a safe area. Due to the urgency of the action this functionality has an average complexity.}

	\item{\textbf{Unlock car}: this functionality receives the position of the user and his ID and the car ID. After that the system proceeds to notify the car to open the doors and calls the \textit{start ride} function. Since it comprises another functionality and communicates with the car we classify this functionality with average complexity.}

	\item{\textbf{Update car}: this functionality requires only two parametres: the car ID and the status to be update. The component involved is the DB. For these reasons this functionality has low complexity.}

	\item{\textbf{Update plug}: as the previous functionality, also this one requires two parametres and affects the DB. So it's a low complexity functionality.}

	\item{\textbf{Set car available}: as the previous functionality, also this one requires two parametres and affects the DB. So it's a low complexity functionality.}

	\item{\textbf{Set car available}: as the previous functionality, also this one requires two parametres and affects the DB. So it's a low complexity functionality.}

	\item{\textbf{Report issue}: this functionality receives a form as input, so a set of strings, and modifies the status of the car in the DB and car itself. So we can consider this functionality as average complexity operation.}
\end{itemize}

	\begin{center}
	\begin{tabular}{ | p{6cm} | p{6cm} | p{6cm} }
		\hline
		\multicolumn{1}{|c|}{\textit{EI}} & \multicolumn{1}{c|}{\textit{Complexity}} & \multicolumn{1}{c|}{\textit{FPs}} \\
		\hline
		Login & Low & 2 \\
		\hline
		User update & Average & 4\\
		\hline
		Pay bill & High & 6x2\\
		\hline
		Create reservation & Average & 4\\
		\hline
		Cancel reservation & Average & 4\\
		\hline
		Start ride & Low & 2\\
		\hline
		End ride & High & 6\\
		\hline
		Park & Average & 4\\
		\hline
		Unlock car & Average & 4\\
		\hline
		Update car & Low & 2\\
		\hline
		Update plugs & Low & 2\\
		\hline
		Set car unavailable & Low & 2\\
		\hline
		Set car available & Low & 2\\
		\hline
		Report issue & Average & 4\\
		\multicolumn{2}{|c|}{\textit{Total}} & \multicolumn{1}{c|}{54}\\
		\hline
	\end{tabular}
\end{center}

\subsubsection{External Inquiries (EQs)}

- get info utente -> low
- get bills -> low
- car search con position -> medium
- car search con address -> medium
- pgs search con position -> medium
- pgs search con address -> medium
- money saving option -> high
- safe area search pos -> medium
- '' add -> medium
- cars in need of maintenance -> low

lowx3 = 3x3 = 9
medium x6 = 4x6 = 24
high x1 = 6x1 = 6

39 per EQs for house music

\subsubsection{External Outputs (EOs)}

- lock car -> low
- unlock car -> low
- richiedi update dalla macchina -> medium

4x2+5x1=13 FPs per EOOOOOOOOO

\subsubsection{Overall estimation}

ILF 73
ELF 15
EI 49
EQ 39
EO 13

Total 189

con AVC 46 per Java EE
SLOC = 189 * 46 = 8694


Forse è più corretto 53? vedi COCOMO II 2000


\subsection{Cost and effort estimation: COCOMO II}


\subsubsection{Scale Drivers}

qua ci vuole la tabella dove ci sono tutti i valori

Precedentedness -> very low 6,20
flexibility -> nominal 3,04
risk -> very high 1,41
team -> very high 1,10
maturity -> level 3 3.12 (non so perchè)

%E=B+0.01x∑ SF,whereB=0.91

%E = 0.91+0.01x(6.20+1.01+1.41+1.1+3.12)-> 12,84 = 1,0384 
rifare il conto, e metterlo sotto effort equation

\subsubsection{Cost Drivers}

RELY -> nominal 1.0 (se l'applicazione non funziona perdiamo soldi e magari qualcuno ci fa causa, ma non succede il fini mondo) al massimo moderate, easily recoverable losses

DATA ->  
dimensioni database: ordine di grandezza database di test 1GB % (meno?)
linee di codice: ~9000
D/P = ~10\^9/10\^4 = 1*10\^5 = 100'000
Very High -> 1.28

CPLEX ->
- control operations:
"Reentrant and recursive coding. Fixed-priority interrupt
handling. Task
synchronization,
complex
callbacks,
heterogeneous
distributed
processing.
Single-
processor hard
real-time control."
Very High

- computational operations
Nominal

- device-dependent operations
Low

- data management operations
High

- User Interface management operations
nominal

(5+3+2+4+3)/5=3.qualcosa

totale Nominal 1.00

RUSE -> none, Low 0.95

DOCU -> right-sized to life cycle needs, Nominal 1.00

TIME -> un numero a caso High 1.11

STOR -> abbiamo un sacco di spazio, Nominal 1.00

PVOL -> major update 2 mesi, High 1.15

ACAP -> Nominal 1.00

PCAP -> capacissimi con java e che talento Very High 0.76

PCON -> no turnover, Very High 0.81

APEX -> poca esperienza, Low 1.10

PLEX -> veeery low, Very Low 1.19

LTEX -> Nominal, 1.00

TOOL -> abbiamo eclipse, Very High 0.78

SITE -> same city or metro area, High 0.93

SCED -> boh, teniamo una schedule non accelerata, Nominal 1.00

\subsubsection{Effort equation}

%E=B+0.01x∑ SF,whereB=0.91

PM = A*(SLOC)\^E * produttoria(cost drivers)
A = 2.94



\subsubsection{Schedule estimation}

TDEV time to develop (in mesi)
TDEV = C * PM\^F
F = (D + 0.2*(E-B))
B = 0.91
C = 3.67
D = 0.28
E = quello della effort equation

% nb no schedule compression
\section{Schedule}


\section{Resource allocation}


\section{Risk Management}

\appendix

\section{Changelog}

\section{Hours of work}

\end{document}
