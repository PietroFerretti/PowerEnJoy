\documentclass[english]{article}
\usepackage[T1]{fontenc}
\usepackage[utf8]{inputenc}
\usepackage{babel}
\usepackage[unicode=true,pdfusetitle,
 bookmarks=true,bookmarksnumbered=false,bookmarksopen=false,
 breaklinks=true,pdfborder={0 0 1},backref=false,colorlinks=false]
 {hyperref}
\usepackage{tabularx}
\usepackage{graphicx}
\graphicspath{{images/}}
\usepackage{svg}
\usepackage{float}
\usepackage{titling}
\renewcommand{\arraystretch}{1.4}
\newcommand{\code}[1]{\texttt{#1}}
\usepackage{array}
\usepackage{caption}

\pretitle{%
	\begin{center}
		\LARGE
		\includegraphics[width=250pt]{../other/Logo_blu.png}\\[\bigskipamount]~\\[\bigskipamount]
	}
\posttitle{\end{center}}

\begin{document}

\title{Politecnico di Milano\\
 A.A. 2016–2017 \\
Software Engineering 2: “PowerEnJoy” \\
\emph{\textbf{Project Plan}}}

\author{Pietro Ferretti, Nicole Gervasoni, Danilo Labanca}
\date{January 21, 2017}
\maketitle

\newpage

\tableofcontents{}

\newpage

\section{Introduction}

\subsection{Purpose}
The purpose of this document is to provide a detailed analysis of the PowerEnjoy software development project in terms of required cost and time. It highlights the estimation of 
\begin{itemize}
\item project size, calculated using the \emph{Function Points approach} by IBM;
\item project cost and effort, calculated using the \emph{COCOMO II} by Boehm.
\end{itemize}
Given the previous information we elaborate a feasible schedule considering all the necessary activities in detail, thus the best resources' allocation on each one. The last section of the document focuses on handling all the possible risks that could be met during the whole process, from the requirements analysis to the final testing and deployment.


\subsection{Scope}
The aim of this project is to specify and design a new digital management software for PowerEnJoy, a car-sharing service that employs electric cars only.

\paragraph{}
PowerEnJoy will offer a very valuable service to its users, letting them borrow cars to drive around the city freely, as an alternative to their own vehicles and public transport.
Among the advantages of using PowerEnJoy we can note being able to find available cars in any place that is served by our system and having dedicated spots to park in (namely, PowerEnJoy's power grid stations).
Furthermore, thanks to the fact that all the cars that we provide are electrically powered, PowerEnJoy is also very environmentally friendly.

%\paragraph{}
%PowerEnJoy's users, after registering, will be able to reserve, unlock and drive the cars our system will provide. Users will be charged per minute until they park the car in a safe area and end the ride.
%Users will be able to park their car temporarily and use it again later, or end their ride remotely.
%
%Our system will incentivize virtuous behaviour by offering several discounts if certain conditions are met (like charging a car at a power grid station).

\newpage
\subsection{List of Definitions and Abbreviations}

\subsubsection{Definitions}

%\begin{itemize}
%%\item{\textit{Guest}: a person that is not registered to the system.}
%\item{\textit{User}: a person that is registered to the system. Users can log in to the system with their email or username and their password. Their first name, last name, date of birth, driving license ID are stored in the database.}
%\item{\textit{Safe area}: a location where the user can park and leave the car. Users can end their ride and park temporarily only in these locations. The set of safe areas is predefined by the system.}
%\item{\textit{Power grid station} or \textit{Charging station}: a place where cars can be parked and plugged in. While a car is plugged in a power grid station its battery will be recharged. Power grid stations are by definition safe areas.}
%\item{\textit{Available car}: a car that is currently not being used by any user, and has not been reserved either. Available cars are in good conditions (not dirty nor damaged) and don't have dead batteries.}
%\item{\textit{Reservation}:
%	\begin{itemize}
%		\item{the operation of making a car reserved for a user, i.e. giving permission to unlock and use the car only for that user, forbidding reservations by other users.}
%		\item{the time period between the moment a reservation is requested and the moment the user unlocks the car, or the reservation is canceled.}
%	\end{itemize}
%}
%\item{\textit{Ride}: the time period from the moment a reserved car is unlocked to the moment the user notifies that he wants to stop using the car and closes all the doors. A ride doesn't stop when a car is temporarily parked, but continues until the user chooses to leave the car definitely.}
%%\item{\textit{Possession}: users that have reserved and unlocked a car are said to have possession of the car. While a user has possession of a car they are the only person that can drive it, lock or unlock it, and no other person can take possession of it until the user frees it. Users lose possession of a car when their ride ends.}
%\item{\textit{Temporary parking}: the act of parking a car in a safe area and, after notifying the system, locking it and leaving it for a finite amount of time. The user that does this retains the right to use the car and can unlock it later to use it again.}
%\item{\textit{Bill}: a record of the money owed by the user at the end of a ride.}
%%\item{\textit{Outstanding bill}: a bill that hasn't been paid yet. }
%\item{\textit{Suspended user}: a user that cannot reserve or use cars. Usually users are suspended because they have outstanding bills that have not been paid.}
%\item{\textit{Payment method}: a way to transfer money from the user to the system. Our system will only accept credit cards and online accounts like Paypal.}
%\item{\textit{Payment API}: an interface to carry out money transactions, offered by the external provider associated to the payment method used (e.g. a bank).}
%\item{\textit{CAN bus}: a vehicle bus standard designed to allow micro controllers and devices to communicate with each other.}
%\end{itemize}

\subsubsection{Acronyms}
\begin{itemize}
\item \textbf{ITPD}: Integration Test Plan Document
\item{\textbf{DD}: Design Document}
\item{\textbf{RASD}: Requirements Analysis and Specification Document}
\item{\textbf{DB}: Database}
%\item{\textbf{CVV}: Card Verification Value}
%\item{\textbf{DOB}: Date of birth}
\item{\textbf{PGS}: Power Grid Station}
\item{\textbf{GPS}: Global Positioning System}
\item{\textbf{API}: Application Programming Interface}
\item{\textbf{ISDTN}: International Standard Date and Time Notation}
\item \textbf{EM}: Effort Multiplier
% anche tutti i cost driver, ecc.?
%\item{\textbf{CAN bus}: Controller Area Network bus}
\item {\textbf{FP}: Function Points}
\item \textbf{ILF}: Internal Logic File
\item \textbf{ELF}: External Logic File
\item \textbf{EI}: External Input
\item \textbf{EO}: External Output
\item \textbf{EQ}: External Inquiries
%\item \textit{DBMS}: Database Management System
%\item \textbf{ETA}: Estimated Time of Arrival
\item \textbf{UI}: User Interface
\end{itemize}

%\subsubsection{Abbreviations}
%\begin{itemize}
%\item{\textbf{[Gx]}: Goal}
%\item{\textbf{[RE.x]}: Functional Requirement}
%\item{\textbf{[UC.x]}: Use Case}
%\end{itemize}

\subsection{List of Reference Documents}

\begin{itemize}
	\item{Requirements analysis and specification document: “RASD.pdf”}
	\item{Design document: “DD.pdf”}
	\item{Integration testing document: “ITPD.pdf”}
	\item{Project description document: “Assignments AA 2016-2017.pdf”}
	\item{Example document: “Project planning example document.pdf”}
	\item{“COCOMO II -- Model Definition Manual”, version 2.1, 1995-2000, Center for Software Engineering, USC} % http://csse.usc.edu/csse/research/COCOMOII/cocomo2000.0/CII_modelman2000.0.pdf
\end{itemize}

\section{Project size, cost and effort estimation}

% descrizione sezione


%% TABELLA
\begin{center}
	\begin{tabular}{ | p{6cm} | p{6cm} | }
		\hline 
		\multicolumn{2}{|c|}{\textbf{Pay a Bill}} \\
		\hline
		\multicolumn{1}{|c|}{\textit{Input}} & \multicolumn{1}{c|}{\textit{Result}} \\
		\hline
		A valid session token, a bill that needs to be paid and a valid payment method &  The transaction is carried out; if it succeeds the bill is marked as paid, otherwise returns failure. \\
		\hline
		A valid session token, a bill that needs to be paid and an ill-formed payment method & An exception is raised. \\
		\hline
		A valid session token and a bill that needs to be paid & The system uses the payment method saved for the user to carry out the transaction; if it succeeds the bill is marked as paid, otherwise returns failure. \\
		\hline
		A valid session token and a bill that has already been paid & An exception is raised. \\
		\hline
		A valid session token and a non-existent bill & An exception is raised. \\
		\hline
		An invalid session token and a bill & An exception is raised (bad authentication). \\
		\hline
	\end{tabular}
\end{center}

% tabella somma FP
\begin{table}[H]
	\centering
	\makebox[\textwidth][c]{
		\begin{tabular}{ |p{8cm}|m{2cm}|p{1cm}| }
			\hline
			\textbf{ELF} & \textbf{Complexity} & \textbf{FPs} \\
			\hline
			elf n1 & Low & 5 \\
			elf n2 & High & 10 \\
			elf n3 & medium & 7 \\
			\hline
			\multicolumn{2}{|l|}{Total} & \textbf{22} \\
			\hline
		\end{tabular}
	}
	\caption{asdfasdf}
\end{table}

% oppure
\begin{table}[H]
	\centering
	\makebox[\textwidth][c]{
		\begin{tabular}{ |p{8cm}|m{2cm}|p{1cm}| }
			\hline
			\textbf{ELF} & \textbf{Complexity} & \textbf{FPs} \\
			\hline
			elf n1 & Low & 5 \\
			elf n2 & High & 10 \\
			elf n3 & medium & 7 \\
			\hline
			\multicolumn{2}{r}{\textit{total}} & \multicolumn{1}{l}{\textbf{22}} \\
		\end{tabular}
	}
	\caption{ewerewr}
\end{table}

% tabella lista cost driver
\begin{table}[H]
	\centering
	\makebox[\textwidth][c]{
		\begin{tabular}{ |p{8cm}|m{3cm}|p{1cm}| }
			\hline
			\textbf{Cost Driver} & \textbf{Rating Level} & \textbf{EM} \\
			\hline
			Documentation match to life-cycle needs (DOCU) & Nominal & 1.00 \\
			\hline
			\multicolumn{2}{|l|}{Total} & \textbf{1.00} \\
			\hline
		\end{tabular}
	}
	\caption{I'm a table. Are u sure?}
\end{table}

\subsection{Size estimation: function points}
Function points are useful in expressing the amount of business functionality our software has to provide to a user and are used to compute an estimation of its size.
After been identified and categorized into one of five types: outputs, inquiries, inputs, internal files, and external interfaces, each functional requirement is then assessed for complexity and assigned a number of function points.\\
We based our computation on tables and values in \emph{COCOMO II Model Definition Manual v. 2.1}.

% tabelline complessità

\subsubsection{Internal Logic Files (ILFs)}
%quali internal logic files abbiamo?
%quali entità di dati abbiamo?

They are all kinds of data used and managed by the application in  order to offer the expected functions.\\
Data will be organize in the following tables in the DB:

\begin{itemize}

	\item \textbf{User} : name, surname, username, password, dob, email, licenseID, cvv, cardNumber, accountStatus

	\item \textbf{Bill} : associatedLicense, total, date, rideID, carID, paymentStatus

	\item \textbf{Car} : model, plate, ID, available, issues

	\item \textbf{Report} :carID, description, associatedLicense, date

	\item \textbf{Safe area} : latitude, longitude, ID

	\item \textbf{PGS} : latitude, longitude, ID

	\item \textbf{Plug} : available, ID

	\item \textbf{Reservation} : ID, associatedLicense, carID, date, status

	\item \textbf{Ride} : ID, associatedLicense, associatedBill, date, status, ridingTime, carID

\end{itemize}

The software will operate directly on the previously listed data and with the tables generated from their relations between each other.\\
All this data are modeled in simple structures so their complexity can be considered low (referring to tables).

\begin{center}
	\begin{tabular}{ |p{8cm}|m{2cm}|p{1cm}| }
		\hline
		\multicolumn{1}{|c|}{\textbf{ILF}} & \multicolumn{1}{c|}{\textbf{Complexity}} & \multicolumn{1}{c|}{\textbf{FPs}} \\
		\hline
		User & Low & 7 \\
		\hline
		Bill & Low & 7\\
		\hline
		Car & Low & 7\\
		\hline
		Report & Low & 7\\
		\hline
		Safe area & Low & 7\\
		\hline
		PGS & Low & 7\\
		\hline
		Plug & Low & 7\\
		\hline
		Reservation & Low & 7\\
		\hline
		Ride & Low & 7\\
		\hline
		\multicolumn{2}{|l|}{\textit{Total}} & \multicolumn{1}{l|}{73} \\
		\hline
	\end{tabular}
\end{center}

\textbf{quel +10 cos'è?}

\begin{center}
$ FPs (ILF) = 7 \times 9 + 10= 73 $
\end{center}

\subsubsection{External Logic Files (ELFs)}

The situations in which our system demands external data is when it needs informations regarding geolocation or when it must guarantee the legal soundness of the Driving License.\\
In particularly:

\begin{itemize}
	\item \textbf{GraphHopper API}:
		\begin{itemize}
			\item{Given the string containing the address, the API returns a pair of float representing the coordinates of that location.}
			\item{Given two pair of coordinates, the API return a float representing the time within two position.}
		\end{itemize}
	\item \textbf{Eucaris API}:
		\begin{itemize}
			\item{Given name, surname, driving license ID and expiration date as string, the API returns a boolean value representing the correspondence with an existing driving license in Eucaris DB.}
		\end{itemize}
\end{itemize}

In the final analysis, as the involved data are string and number with restrained size, we can assess this logic files as low complexity.\\

\begin{center}
	\begin{tabular}{ |p{8cm}|m{2cm}|p{1cm}| }
		\hline
		\multicolumn{1}{|c|}{\textbf{ELF}} & \multicolumn{1}{c|}{\textbf{Complexity}} & \multicolumn{1}{c|}{\textbf{FPs}} \\
		\hline
		Reverse geocoding & Low & 5 \\
		\hline
		Isochrone distance & Low & 5\\
		\hline
		Driving Licenes legal soudness & Low & 5\\
		\hline
		\multicolumn{2}{|l|}{\textit{Total}} & \multicolumn{1}{l|}{15} \\
		\hline
	\end{tabular}
\end{center}

\subsubsection{External Inputs (EIs)}

PowerEnjoy offers a remarkable series of functionalities that required user's input.\\
In particularly:

\begin{itemize}

	\item{\textbf{Login}: this functionality demands only two strings as parameters, the username and the password, that will be compared with the ones stored in the DB. We can consider as a low complexity operation.}
	
	\item{\textbf{User update}: this functionality includes a collection of operations that allow to modify each aspect of user's profile. The input data are simple strings. Since the possibility are conspicuous and the different elaborations aren't basic and futhermore they interest several components, we can regard this functionality as an average complexity operation.}
	
	\item{\textbf{Pay bill} (automatically/manually): this is one of the most complex operations. It involves internal components and external APIs and it demands two numbers as input. Given the relevance of the operation and the parts interested, we can consider this as high complexity operation.}

	\item{\textbf{Create reservation}: this functionality requires as input the user ID and car ID, that we can consider simple inputs, while his complexity is due to the components involved. In fact at the initial moment of the creation, the application verifies if the user is suspended and after that is modified the availabilty field of the tuple representing the car and the car itself is put in order to be unlocked. Moreover it's inserted in the list of reservation made by the user this new one. Given that we classify the operation as having average complexity.}

	\item{\textbf{Cancel reservation}: the analysis made for the previous functionality is valid also for this one. The operations are comparable as elaboration and thanks to this the complexity of this functionality is average.}

	\item{\textbf{Start ride}: this action is very simple as it's demands only the user ID and the car ID and it tallies the duration of the ride. Made this consideration we assues that this functionality as low complexity.}

	\item{\textbf{End ride}: this functionality is the most complex one because involves various components and includes the payment functionality that has high complexity. When a ride ends the car is set as available in the DB, it creates a bill for the user and initiates the payment. As usual the inputs are simple strings.}

	\item{\textbf{Park}: this functionality expects as inputs the position of the car that are a pair of numbers, the user ID and the car ID. In the elaboration of this data the components interested are the internal car system and the DB to verify if the position belongs to a safe area. Due to the urgency of the action this functionality has an average complexity.}

	\item{\textbf{Unlock car}: this functionality receives the position of the user and his ID and the car ID. After that the system proceeds to notify the car to open the doors and calls the \textit{start ride} function. Since it comprises another functionality and communicates with the car we classify this functionality with average complexity.}

	\item{\textbf{Update car}: this functionality requires only two parametres: the car ID and the status to be update. The component involved is the DB. For these reasons this functionality has low complexity.}

	\item{\textbf{Update plug}: as the previous functionality, also this one requires two parametres and affects the DB. So it's a low complexity functionality.}

	\item{\textbf{Set car available}: as the previous functionality, also this one requires two parametres and affects the DB. So it's a low complexity functionality.}

	\item{\textbf{Set car available}: as the previous functionality, also this one requires two parametres and affects the DB. So it's a low complexity functionality.}

	\item{\textbf{Report issue}: this functionality receives a form as input, so a set of strings, and modifies the status of the car in the DB and car itself. So we can consider this functionality as average complexity operation.}
\end{itemize}

	\begin{center}
	\begin{tabular}{ |p{8cm}|m{2cm}|p{1cm}| }
		\hline
		\multicolumn{1}{|c|}{\textbf{EI}} & \multicolumn{1}{c|}{\textbf{Complexity}} & \multicolumn{1}{c|}{\textbf{FPs}} \\
		\hline
		Login & Low & 3 \\
		\hline
		User update & Average & 4\\
		\hline
		Pay bill & High & 6x2\\
		\hline
		Create reservation & Average & 4\\
		\hline
		Cancel reservation & Average & 4\\
		\hline
		Start ride & Low & 3\\
		\hline
		End ride & High & 6\\
		\hline
		Park & Average & 4\\
		\hline
		Unlock car & Average & 4\\
		\hline
		Update car & Low & 3\\
		\hline
		Update plugs & Low & 3\\
		\hline
		Set car unavailable & Low & 3\\
		\hline
		Set car available & Low & 3\\
		\hline
		Report issue & Average & 4\\
		\hline
		\multicolumn{2}{|l|}{\textit{Total}} & \multicolumn{1}{l|}{60}\\
		\hline
	\end{tabular}
\end{center}

\subsubsection{External Inquiries (EQs)}

In this section we will discuss about External Inquiries that could be defined as elementary processes that send data or control information outside the application boundary.; the primary intent of an external inquiry is to present information to a user through the retrieval of data or control information from an ILF of EIF. (copiata)\\
In our case we have: 

\begin{itemize}
	\item{\textbf{Get user info}: it's returned a set of strings representing the user's info, from the DB.}

	\item{\textbf{Get bills}: it's returned a list of bills, made by strings, of the user, from the DB.}

	\item{\textbf{Car/PGS/Safe area search with position}: in this case the function needs a parameter that is a pair of number and returns a list extracted from the DB. For these operations there are some elaborations to do and because of that we classify these actions having average complexity.}

	\item{\textbf{Car/PGS/Safe area search with address}: as for the previous cases, also these functionalities have an average complexity even though there is an interaction with the \textit{GraphHopper API} in order to convert the address in coordinates.}

	\item{\textbf{Money Saving Option}: this is the most complex operation because it should extract informations from the DB and elaborate them to find the best solution.}

	\item{\textbf{Cars in need of maintenance}: this is a simple functionality that is employed by maintenance operators and returns a list of cars that are unavailable.}
\end{itemize}


\begin{center}
	\begin{tabular}{ |p{8cm}|m{2cm}|p{1cm}| }
		\hline
		\multicolumn{1}{|c|}{\textbf{EQ}} & \multicolumn{1}{c|}{\textbf{Complexity}} & \multicolumn{1}{c|}{\textbf{FPs}} \\
		\hline
		Get user info & Low & 3 \\
		\hline
		Get bills & Low & 3\\
		\hline
		Car/PGS/Safe area search with position & Average & 4x3\\
		\hline
		Car/PGS/Safe area search with address & Average & 4x3\\
		\hline
		Money Saving Option & High & 6\\
		\hline
		Cars in need of maintenance & Low & 3 \\
		\hline
		\multicolumn{2}{|l|}{\textit{Total}} & \multicolumn{1}{l|}{39}\\
		\hline
	\end{tabular}
\end{center}

\subsubsection{External Outputs (EOs)}

The situation our system needs to notify external agents are the following:

\begin{itemize}
	\item{\textbf{Un/Lock car}}

	\item{\textbf{Car update}}
\end{itemize}

The first case has a low complexity while the second is a little bit more complex because it needs to extract the information from the car.

\begin{center}
	\begin{tabular}{ |p{8cm}|m{2cm}|p{1cm}| }
		\hline
		\multicolumn{1}{|c|}{\textbf{EO}} & \multicolumn{1}{c|}{\textbf{Complexity}} & \multicolumn{1}{c|}{\textbf{FPs}} \\
		\hline
		Un/Lock car & Low & 4x2 \\
		\hline
		Car update & Average & 5\\
		\hline
		\multicolumn{2}{|l|}{\textit{Total}} & \multicolumn{1}{l|}{13}\\
		\hline
	\end{tabular}
\end{center}

\subsubsection{Overall estimation}

\\
\begin{center}
	\begin{tabular}{|p{5cm}|p{1cm}|}
		\hline
		\textbf{Function Type} & \textbf{FPs} \\
		\hline
		Internal Logic Files & 73 \\
		External Logic Files & 15 \\
		External Inputs & 60 \\
		External Inquiries & 39 \\
		External Outputs & 13 \\
		\hline
		\textit{Total} & 200 \\
		\hline
	\end{tabular}
\end{center}

This table contains a recap of the evaluations of each type of function point.\\
Thanks to estimation of the function points we can appraisal the approximately number of lines of code: since that our system is developed using Java EE, we can extrapolate the AVC factor from this table\footnote{\href{http://www.qsm.com/resources/function-point-languages-table}{http://www.qsm.com/resources/function-point-languages-table}}. 

\begin{center}
$SLOC = AVC \times number of function points$
\end{center}

where

$ AVC $ is equale to $46$
$ number of function points$ is equal to $200$

So at the end, we have:

\begin{center}
$SLOC = 9200$
\end{center}

\subsection{Cost and effort estimation: COCOMO II}

We use the COCOMO II Model for cost and effort estimation.
*cos'è il COCOMO II?*

We use the Post-Architecture Model, because the architecture and everything has already been described in depth.

*qualche dettaglio*

\subsubsection{Scale Factors}

*what are scale factors*

qua ci vuole la tabella dove ci sono tutti i valori

% rate, judge, classify, assess, consider

Scale factors:
\begin{itemize}
	\item \textbf{Precedentedness (PRED):} "If a product is similar to several previously developed projects, then the precedentedness is high."
	how similar is this product to previously developed one
	how many similar product have we developed already
	We never developed anything similar [and on this scale], except for the client-server architecture. We rate this factor as Very Low.
	\item \textbf{Development Flexibility (FLEX):} "Need for software conformance with pre-established requirements \& external interface specifications, + premium on early completion"
	We have to 'considerably' follow the requirements we specified to reach the customer's goals. We rate this as Nominal.
	\item \textbf{Architecture / Risk Resolution (RESL):} High if everything is well planned, the risks have all been identified and accounted for, the architecture has been carefully designed, low uncertainty and low risk in general.
	our architecture is good
	our risk management plan is (we consider it) thorough
	We rate this as High.
	\item \textbf{Team Cohesion (TEAM):} experience as a team, willingness to work together, shared vision and commitments.
	We are great at this, we rate this as Very High.
	\item \textbf{Process Maturity (PMAT):} follows the Capability Maturity Model (CMM).
	nb:
	level 1 -> inital, uncontrolled
	level 2 -> managed
	level 3 -> defined
	level 4 -> quantitatively managed
	level 5 -> optimized
	we are not at a level where everything is accounted for, but we do follow a clear process
	Around level 3, equivalent to a High.
\end{itemize}

\begin{center}
	\begin{tabular}{|p{6cm}|p{2cm}|p{1cm}|}
		\hline
		\multicolumn{1}{|c|}{\textbf{Scale Factor}} & \multicolumn{1}{c|}{\textbf{Level}} & \multicolumn{1}{c|}{\textbf{Value}} \\
		\hline
		Precedentedness (PREC) & Very Low & 6.20 \\
		Development Flexibility (FLEX) & Nominal & 3.04 \\
		Architecture / Risk Resolution (RESL) & High & 2.83 \\
		Team Cohesion (TEAM) & Very High & 1.10 \\
		Process Maturity (PMAT) & High & 3.12 \\
		\hline
		\multicolumn{2}{|l|}{\textit{Total}} & 16.29 \\
		\hline
	\end{tabular}
\end{center}

\subsubsection{Cost Drivers}

*what are cost drivers*
each of these factors can increase or decrease the amount of effort needed to develop the software

We follow the tables in the COCOMO II document.

% dividere i cost drivers per categorie di fattori?

Cost Drivers:
\begin{itemize}

\item \textbf{Required Software Reliability (RELY):} This is the measure of the extent to which the software must perform its intended function over a period of time. If the effect of a software failure is only slight inconvenience then RELY is very low. If a failure would risk human life then RELY is very high.

Even if our system doesn't work the customers have alternatives. And we implement safeguards in the cars not to get stuck inside or anything else. Our software doesn't have any way to control the way the cars drive.
Moderate, easily recoverable losses.

All in all, \textit{Nominal}, 1.00

\item \textbf{Database Size (DATA):} This cost driver attempts to capture the effect large test data requirements have on product development. The rating is determined by calculating D/P, the ratio of bytes in the testing database to SLOC in the program.

We *estimate* our testing database to be around 1GB of size. Given that we estimated the source code to be around ~9000 lines long, this brings the D/P ratio to a order of magnitude of $10^5$.

\textit{Very High}, 1.28 

%DATA ->  
%dimensioni database: ordine di grandezza database di test 1GB % (meno?)
%linee di codice: ~9000
%D/P = ~10\^9/10\^4 = 1*10\^5 = 100'000
%Very High -> 1.28

\item \textbf{Product Complexity (CPLX):} Complexity is divided into five areas: control operations, computational operations,
device-dependent operations, data management operations, and user interface management operations.

Checking on the COCOMO tables for product complexity and averaging the results we judge the complexity rating to be Nominal on average.

\textit{Nominal}, 1.00

%CPLEX ->
%- control operations:
%Very High
%- computational operations
%Nominal
%- device-dependent operations
%Low
%- data management operations
%High
%- User Interface management operations
%nominal
%
%(5+3+2+4+3)/5=3.qualcosa
%totale Nominal 1.00

\item \textbf{Developed for Reusability (RUSE):} This cost driver accounts for the additional effort needed to construct components
intended for reuse on current or future projects.

We have no requirements asking us to develop code ready to be reused in other projects or products. No reusability constraints.

\textit{Low}, 0.95

%RUSE -> none, not in requirements, Low 0.95

\item \textbf{Documentation Match to Life-Cycle Needs (DOCU):} In COCOMO II, the rating scale for the DOCU cost driver is evaluated in terms of the suitability of the project’s documentation to its life-cycle needs.

We have no specific instructions about documentation, so we're going standard; documentation just right for the product lye-cycle needs.

\textit{Nominal}, 1.00

%DOCU -> right-sized to life cycle needs, Nominal 1.00

\item \textbf{Execution Time Constraint (TIME):} This is a measure of the execution time constraint imposed upon a software system. The rating is expressed in terms of the percentage of available execution time expected to be used by the system or subsystem consuming the execution time resource.

Our software will be heavy on performance? Or more like "hardware is expensive, let's keep it active at 70\% all the time on average". No waste of processing time.

\textit{High}, 1.11

%TIME -> un numero a caso High 1.11

\item \textbf{Main Storage Constraint (STOR):} This rating represents the degree of main storage constraint imposed on a software system or subsystem.

Storage is really, really cheap today. We're not in 2000 anymore.
We can buy 1000TB of hard disk and be on the safe side easy.

\textit{Nominal}, 1.00

%STOR -> abbiamo un sacco di spazio, Nominal 1.00

\item \textbf{Platform Volatility (PVOL):} “Platform” is used here to mean the complex of hardware and software (OS, DBMS, etc.)
the software product calls on to perform its tasks. This rating ranges from low, where there is a major change every 12 months, to very high, where there is a major change every two weeks.

We want our software to be kept updated to fix bugs and possible vulnerabilities. A major update every 2 months and minor ones every week seems reasonable.

\textit{High}, 1.15

%PVOL -> major update 2 mesi, High 1.15

\item \textbf{Analyst Capability (ACAP):} Analysts are personnel who work on requirements, high-level design and detailed design. The major attributes that should be considered in this rating are analysis and design ability, efficiency and thoroughness, and the ability to communicate and cooperate.

We are average analysts, nothing special as abilities.

\textit{Nominal}, 1.00
%ACAP -> Nominal 1.00

\item \textbf{Programmer Capability (PCAP):} Major factors which should be considered in the rating are ability, efficiency and thoroughness, and the ability to communicate and cooperate. also capability not experience

We are very confident in our abilities in programming, great talents. And, we can work as a team

\textit{Very High}, 0.76

%PCAP -> capacissimi con java e che talento Very High 0.76

\item \textbf{Personnel Continuity (PCON):} The rating scale for PCON is in terms of the project’s annual personnel turnover: from
3\%, very high continuity, to 48\%, very low continuity

(Not sure if this is correct) We are going to be the ones developing this piece of software, and we are pretty dedicated to following this through until the end. We rate the probability of any of us leaving as negligible.

\textit{Very High}, 0.81

%PCON -> no turnover, Very High 0.81

\item \textbf{Applications Experience (APEX):} The rating for this cost driver is dependent on the level of applications experience of the project team developing the software system or subsystem. The ratings are defined in terms of the project team’s equivalent level of experience with this type of application.

We don't have much experience with this type of application, except for some basic client-server architecture.

\textit{Low}, 1.10

%APEX -> poca esperienza, Low 1.10

\item \textbf{Platform Experience (PLEX):} Understanding the use of more powerful platforms, including more graphic user interface, database, networking, and distributed middleware capabilities.

We have close to zero experience with database, networking, and distributed middleware software.

\textit{Very Low}, 1.19

%PLEX -> veeery low, Very Low 1.19

\item \textbf{Language and Tool Experience (LTEX):} This is a measure of the level of programming language and software tool experience of the project team developing the software system or subsystem.

We have experience with Java, and a little JEE too. Good enough, rating this as Nominal (not high, not low).

\textit{Nominal}, 1.00

%LTEX -> Nominal, 1.00

\item \textbf{Use of Software Tools (TOOL):} What are the capabilities of the tools and the editors we are going to use.

Modern Java IDEs like Eclipse and Netbeans can do almost anything: basically "strong, mature, proactive life-cycle tools, well integrated with processes, methods, reuse". 
git integration, sonar, hints ecc.

\textit{Very High}, 0.78

%TOOL -> abbiamo eclipse, Very High 0.78

\item \textbf{Multisite Development (SITE):} averaging of two factors: site collocation (from fully collocated to international distribution) and communication support (from surface mail and some phone access to full interactive multimedia)

"The developers"/We are located more or less in the same metro area, and are able to meet whenever to solve problems.
Also communication is not a problem (whatsapp, ecc.)
+ Trello (tools for organization and issue distribution non so)

\textit{High}, 0.93
%SITE -> same city or metro area, High 0.93

\item \textbf{Required Development Schedule (SCED):} This rating measures the schedule constraint imposed on the project team developing the software. The ratings are defined in terms of the percentage of schedule stretch-out or acceleration with respect to a nominal schedule for a project requiring a given amount of effort.

We have no constraint on time or schedules. We can just keep the spontaneous time required for the development, with no schedule acceleration or stretching. (100\%)

\textit{Nominal}, 1.00

%SCED -> boh, teniamo una schedule non accelerata, Nominal 1.00

\end{itemize}

\begin{center}
	\begin{tabular}{|p{8cm}|p{2cm}|p{1cm}|}
		\hline
		\multicolumn{1}{|c|}{\textbf{Cost Driver}} & \multicolumn{1}{c|}{\textbf{Level}} & \multicolumn{1}{c|}{\textbf{Value}} \\
		\hline
		Required Software Reliability (RELY) & Nominal & 1.00 \\
		Database Size (DATA) & Very High & 1.28 \\
		Product Complexity (CPLX) & Nominal & 1.00 \\
		Required Reusability (RUSE) & Low & 0.95 \\
		Documentation match to lyfe-cycle needs (DOCU) & Nominal & 1.00 \\
		Execution Time Constraint (TIME) & High & 1.11 \\
		Main Storage Constraint (STOR) & Nominal & 1.00 \\
		Platform Volatility (PVOL) & High & 1.15 \\
		Analyst Capability (ACAP) & Nominal & 1.00 \\
		Programmer Capability (PCAP) & Very High & 0.76 \\
		Personnel Continuity (PCON) & Very Low & 0.81 \\
		Application Experience(APEX) & Low & 1.10 \\
		Platform Experience (PLEX) & Very Low & 1.19 \\
		Language and Tool Experience (LTEX) & Nominal & 1.00 \\
		Usage of Software Tools (TOOL) & Very High & 0.78 \\
		Multisite Development (SITE) & High & 0.93 \\
		Required Development Schedule (SCED) & Nominal & 1.00 \\
		\hline
		\multicolumn{2}{|l|}{\textit{Total}} & 0.907 \\
		\hline
	\end{tabular}
\end{center}


\subsubsection{Effort equation}

We use the formula for the Post-Architecture Model

$PM$ is the estimated Person-Month effort needed to develop the code.


\begin{center}
$ PM = A \times Size^E \times \prod_{i=1}^{17} EM_i $
\end{center}

\begin{center}
$ E = B + 0.01 \times \sum_{j=1}^{5} SF_j $
\end{center}
where

$ A = 2.94 $ is a calibrated constant 

$ B = 0.91 $ is a calibrated constant

$EM_i$ are the effort multipliers (cost drivers)

$SF_j$ are the scale factors

$Size$ is the estimated thousands of lines of source code (KSLOC)


scale drivers = 6.20 + 3.04 + 1.41 + 1.10 + 3.12 = 14.87
cost drivers (effort multipliers) = 1.0*1.28*1.00*0.95*1.00*1.11*1.00*1.15*1.00*0.76*0.81*1.10*1.19*1.00*0.78*0.93*1.00 = 0.907\\
then

$E = 0.91 + 0.01 \times 16.29 = 1.07$

$PM = 2.94 \times 8.694^{1.06} \times 0.907 = 26.97$ person-months

\subsubsection{Schedule estimation}

$TDEV$ is the time necessary to develop the code, in calendar months

assuming no schedule compression, no stretching-out

\begin{center}
	$TDEV = C \times PM^F$
\end{center}

\begin{center}
	$F = D + 0.2 \times (E-B)$
\end{center}
where

$B = 0.91$ is a calibrated constant

$C = 3.67$ is a calibrated constant

$D = 0.28$ is a calibrated constant

$E$ is the scaling exponent for the effort equation

$PM$ person-months obtained in the effort estimation\\
then

$F = 0.28 + 0.2 \times (1.07 - 0.91) = 0.31$

$TDEV = 3.67 \times 26.97^{0.31} = 10.19$ months

\section{Schedule}


\section{Resource allocation}


\section{Risk Management}
Project managers assess and monitor risks that may affect a project, taking action if needed.   
\begin{center}
	\begin{tabular}{|c|c|c|}
		\hline
		\textbf{Risk} & \textbf{Probability } &  \textbf{Impact} \\
		\hline
		Personnel shortfall & moderate & catastrophic\\(recruitment issues, employee illness or accidents, ...) & & \\
		\hline
		Inaccurate Requirements & moderate & critical\\
		\hline
		Unrealistic Schedule & moderate & critical\\
		\hline
		Unrealistic Budget & moderate & critical\\
		\hline
		Stakeholder commitment loss & low & critical\\
		\hline
		New car rental laws & low & catastrophic \\
		\hline
		Inability to obtain proper permits from authorities  & low & catastrophic \\
		\hline
		Inability to obtain deal with mobile data provider  & low & critical \\
		\hline		
		Issues with hardware supplier & moderate & critical \\
		(wrong/defected items, late deliveries, ...) & & \\
		%nfrastruttura? 
		%poco successo col pubblico non 
		\hline
		Wrong user interface & moderate & critical\\
		\hline
	\end{tabular}
\end{center}


\appendix

\section{Changelog}

\section{Hours of work}

\end{document}
