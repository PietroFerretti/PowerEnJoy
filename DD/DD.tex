\documentclass[english]{article}
\usepackage[T1]{fontenc}
\usepackage[utf8]{inputenc}
\usepackage{babel}
\usepackage[unicode=true,pdfusetitle,
 bookmarks=true,bookmarksnumbered=false,bookmarksopen=false,
 breaklinks=true,pdfborder={0 0 1},backref=false,colorlinks=false]
 {hyperref}
\usepackage{tabularx}
\usepackage{graphicx}
\graphicspath{{images/}}
\usepackage{svg}
\usepackage{float}
\usepackage{titling}
\renewcommand{\arraystretch}{1.4}

\pretitle{%
	\begin{center}
		\LARGE
		\includegraphics[width=250pt]{../other/Logo_blu.png}\\[\bigskipamount]~\\[\bigskipamount]
	}
\posttitle{\end{center}}

\begin{document}

\title{Politecnico di Milano\\
 A.A. 2016–2017 \\
Software Engineering 2: “PowerEnJoy” \\
\emph{Design Document}}

\author{Pietro Ferretti, Nicole Gervasoni, Danilo Labanca}
%\date{December 11, 2016}
\maketitle

\newpage

\tableofcontents{}

\newpage

\section{Introduction}

\subsection{Purpose}

\paragraph{}
"The purpose of this document is to provide a complete and detailed description and specification of a digital management system for PowerEnJoy, an electric car sharing service. This document will
illustrate functional and non-functional requirements of the software to-be, outlining costraints, showing potential user interfaces for the software, and explaining the domain assumptions made. Additionally, at the end of the document we will present an Alloy model to further specificate the world and environment our system will have to manage."

\paragraph{}
"This document is first and foremost a description of the project for all interested stakeholders and a reference for future development, but can also be used as a basis for legally binding agreements."


\subsection{Scope}

\paragraph{}
text here

\paragraph{}
more text here

\paragraph{}
....

\newpage
\subsection{Definitions, Acronyms, Abbreviations}

\subsubsection{Definitions}

\begin{itemize}
\item{\textit{Guest}: a person that is not registered to the system.}
\item{\textit{User}: a person that is registered to the system. Users can log in to the system with their email or username and their password. Their first name, last name, date of birth, driving license ID are stored in the database.}
\item{\textit{Safe area}: a location where the user can park and leave the car. Users can end their ride and park temporarily only in these locations. The set of safe areas is predefined by the system.}
\item{\textit{Power grid station}: a place where cars can be parked and plugged in. While a car is plugged in a power grid station its battery will be recharged. Power grid stations are by definition safe areas.}
\item{\textit{Available car}: a car that is currently not being used by any user, and has not been reserved either. Available cars are in good conditions (not dirty nor damaged) and don’t have dead batteries.}
\item{\textit{Reservation}:
	\begin{itemize}
		\item{the operation of making a car reserved for a user, i.e. giving permission to unlock and use the car only for that user, forbidding reservations by other users.}
		\item{the time period between the moment a reservation is requested and the moment the user unlocks the car, or the reservation is canceled.}
	\end{itemize}
}
\item{\textit{Ride}: the time period from the moment a reserved car is unlocked to the moment the user notifies that he wants to stop using the car and closes all the doors. A ride doesn’t stop when a car is temporarily parked, but continues until the user chooses to leave the car definitely.}
\item{\textit{Possession}: users that have reserved and unlocked a car are said to have possession of the car. While a user has possession of a car they are the only person that can drive it, lock or unlock it, and no other person can take possession of it until the user frees it. Users lose possession of a car when their ride ends.}
\item{\textit{Temporary parking}: the act of parking a car in a safe area and, after notifying the system, locking it and leaving it for a finite amount of time. The user that does this retains the right to use the car and can unlock it later to use it again.}
\item{\textit{Bill}: a record of the money owed by the user at the end of a ride.}
\item{\textit{Outstanding bill}: a bill that hasn’t been paid yet. }
\item{\textit{Suspended user}: a user that cannot reserve or use cars. Usually users are suspended because they have outstanding bills.}
\item{\textit{Payment method}: a way to transfer money from the user to the system. Our system will only accept credit cards and online accounts like Paypal.}
\item{\textit{Payment API}: an interface to carry out money transactions, offered by the external provider associated to the payment method used (e.g. a bank).}
\item{\textit{CAN bus}: a vehicle bus standard designed to allow micro controllers and devices to communicate with each other.}
\end{itemize}

\subsubsection{Acronyms}
\begin{itemize}
\item{\textbf{RASD}: Requirements Analysis and Specification Document}
\item{\textbf{DB}: Database}
\item{\textbf{CVV}: Card Verification Value}
\item{\textbf{DOB}: Date of birth}
\item{\textbf{PGS}: Power Grid Station}
\item{\textbf{GPS}: Global Positioning System}
\item{\textbf{CAN bus}: Controller Area Network bus}
\end{itemize}

\subsubsection{Abbreviations}
\begin{itemize}
\item{\textbf{[Gx]}: Goal}
\item{\textbf{[RE.x]}: Functional Requirement}
\item{\textbf{[UC.x]}: Use Case}
\end{itemize}

\subsection{Reference Documents}

\begin{itemize}
	\item{ISO/IEC/IEEE Std. 29148:2011, “Systems and software engineering -- Life cycle processes -- Requirements engineering”}
	\item{Specification document: “Assignments AA 2016-2017.pdf”}
\end{itemize}

\newpage{}

\subsection{Document Structure}
This document is structured as follows:
\begin{itemize}
\item{\textbf{Section 1 -- Introduction:} "it is a presentation of the document and the product; it contains the information needed to understand the whole document."}
\item{\textbf{Section 2 -- Architectural Design:} asdf}
\item{\textbf{Section 3 -- Algorithm Design:} qewr}
\item{\textbf{Section 4 -- User Interface Design:} }
\item{\textbf{Section 5 -- Requirements Traceability:} }
\item{\textbf{Effort Spent} }
\item{\textbf{References} }
\item{\textbf{Revisions} }
\end{itemize}

\newpage

\section{Architectural Design}

\subsection{Overview}

\subsection{Component View}

\subsection{Deployment View}

\subsection{Runtime View}

\subsection{Component Interfaces}

\subsection{Selected Architectural Styles and Patterns}

\subsection{Other Design Decisions}

\newpage

\section{Algorithm Design}

\newpage

\section{User Interface Design}

\newpage

\section{Requirements Traceability}

\newpage

\section{Effort Spent}

\newpage

\section{Revisions}

\subsection{Changelog}
\begin{itemize}
  \item{}
\end{itemize}
\end{document}
